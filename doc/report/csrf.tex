\label{CSRFattack}
\section{CSRF Attack}


Cross Site Request Forgery (CSRF or XSRF) works by injecting network querries to the victim's web browser.  By doings so
the attacker is able to hijack the user's session, exploiting the server's trust to that user.  The server would be fooled
that the attacker is a legitimate user and allow him to change the session state or give him access to sensitive data.
The CSRF could also be used to exploit the user's trust to a server, by logging the victim under the attacker's, without 
the first noticing, and logging hist activity or stealing sensitive data the victim may trie to send through querries.
In this section we will see a possible scenarios of how such an attack could be launched, in order to better understand
how it works, and define the prerequirements needed to successfully deliver the attack. 

\subsection{Example}
A user usually interacts with a website by sending requests to a server in a form of a url including parameter values.  
The attacker simply needs to fool the victim into sending a request url with his parameter values.  For example, consider
the following GET request: 
\begin{lstlisting}
GET 
http://bank.com/transfer.do?\
acct=Bob&amount=1000\ 
HTTP/1.1
\end{lstlisting}
Sending this GET request a user can transfer money from his account to another user's account (in our example Bob).
In order to launch a CSRF attack, an attacker needs to generate his own url in the exact fashion a legitimate url
would be generated. For example a forged url could be the following
\footnote{The example was taken from \cite{website:owasp-csrf}}
, where our attacker's account is Alice:
\begin{lstlisting}
http://bank.com/transfer.do?\
acct=Alice&amount=1000\ 
HTTP/1.1
\end{lstlisting}
All the attacker need to do now is make the user sent a GET request using his url.  A simple scenario would be to
the link through an email or a fraud website, where the user is tricked to click on the link.  Ofcourse, in order
to successfully complete the transaction the victim should already be logged in, so that the browser will include
the cookies that manage the state of the session between the server and the victim.  There are even more 
sophisticated ways to force the user's web browser to send the request, without requiring any action from the 
victim other than opening a website or an email.  This is accmplished by putting the url in an html IMAGE, SCRIPT 
or IFRAME tag src attribute, in which case the web browser will automatically load the elements, and make an GET 
request for the resource, where in fact he will be senting the request url.  Note that the POST request method
is not CSRF-proof, since an attacker could include a form with hidden values to generate a POST request with 
all the necessary parameters.

\subsection{Attack Prerequirements}
Understanding the above example, we can deduct that following prerequirements are needed for a CSRF attack to be
successful:
\begin{enumerate}
	\item The attacker knows how a valid request url is formed 
	\item The attacker can provide correct values for all the paremeters
	\item The attacker need to lure the victim into a fraud website or email, while the victim is logged in the target website.
\end{enumerate}    
By observing the above we can see how such an attack could be stopped.  Either the server or the browser needs to know
where did the request originated from, or make it impossible for the attacker to guess the correct values for all the
parameters, by using some secret between user and server. 
